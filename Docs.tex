\documentclass[a4paper, 12pt]{article}
\usepackage[top=2cm, bottom=2cm, left=2.5cm, right=2.5cm]{geometry}
\usepackage[utf8]{inputenc}
\usepackage{amsmath, amsfonts, amssymb}
\DeclareMathOperator{\sen}{sen}
\newcommand{\limite}{\displaystyle\lim}

\begin{document}

% Página de Título
\begin{titlepage}
    \begin{center}
        \vspace*{1cm}
        
        % Título
        \textbf{\LARGE Documentação \LaTeX}
        
        \vspace{0.5cm}
        
        % Descrição
        Esta é uma documentação básica que demonstra como lidar com algumas operações básicas em \LaTeX. Este documento será atualizado conforme a complexidade do conteúdo aumentar.
        
        \vfill
        
        % Linha Horizontal
        \hrulefill
        
        \vfill
        
        % Sumário
        \textbf{\large Sumário}
        
        % Lista Enumerada
        \begin{enumerate}
            \item \underline{Operações Aritméticas}
            \item \underline{Equação Polinomial do 2° Grau}
            \item \underline{Conjuntos Numéricos}
            \item \underline{Funções}
            \item \underline{Cálculo}
        \end{enumerate}
        
        \vfill
        
        % Linha Horizontal
        \hrulefill
        
    \end{center}
\end{titlepage}

% Nova Página
\newpage

% Capítulo 1
\section*{Capítulo 1: Operações Aritméticas}

\begin{enumerate}
    \item \textbf{Operações Aritméticas Básicas}
    \begin{enumerate}
        \item $a + b$
        \item $a - b$
        \item $a \cdot b$
        \item $a \times b$
        \item $a \div b$
        \item Isso é uma fração $\dfrac{a}{b}$
        \item Raízes $\sqrt[3]{2}$
        \item Expoentes $a^b$
        \item Expoentes $a^{b \div c}$
        \item Índices $a_1$
    \end{enumerate}
    
\end{enumerate}

% Capítulo 2
\section*{Capítulo 2: Equação Polinomial do 2° Grau}

% Texto Explicativo
\begin{flushleft}
    Uma equação da forma $ax^2 + bx + c = 0$, onde $a \neq 0$, será chamada de equação polinomial do 2° grau.
    A solução dessa equação é dada por:
\end{flushleft}

% Fórmula Matemática

\[x = \frac{-b \pm \sqrt{b^2 - 4ac}}{2a}.\]

\section*{Capítulo 3: Conjuntos Numéricos}

\begin{enumerate}
    \item Sejam os conjuntos $A = \{1;\ 2;\ 3;\ 4\}$,
    $B = \{x \in \mathbb{Z}\, |\, -2 \leq x < 4\}$ e 
    $C = \{x \in \mathbb{N} \, |\, x \geq 2\}$. Responda aos itens abaixo.

    \begin{enumerate}
        \item $A \cap B$
        \item $B \cup C$
        \item $A - C$
        \item $C\setminus B$
    \end{enumerate}

    \item Classifique em verdadeiro ou falso.
    \begin{enumerate}
        \item $\mathbb{Z} \subset \mathbb{N}$
        \item $\mathbb{R} \subset \mathbb{Q}$
        \item $\mathbb{Z} \not\subset \mathbb{N}$
        \item $\mathbb{Z} \not\supset \mathbb{N}$
        \item $0 \not\in \mathbb{R}\setminus\mathbb{Q}$
        \item $\forall x \mathbb{N}$, temos $x\geq 0$
        \item $\exists x \in \mathbb{R}$, tal que $\sqrt{x}\not\in\mathbb{R}$.
        \item $7 \not\in \{x\in \mathbb{N} \, |\, \textrm{é par}\}$
    \end{enumerate}
\end{enumerate}

\section*{Capítulo 4: Funções}

\begin{enumerate}
    \item \textbf{Função Quadrática:}
    
    Seja a função \(f:\mathbb{R} \to \mathbb{R}\) definida por 
    \[f(x) = \frac{1}{2} x^2 - 2x + 1.\]
    
    \begin{enumerate}
        \item Esboce o gráfico da função.
    \end{enumerate}
    
    \item \textbf{Função por Cláusulas:}
    
    Seja a função \(f(x)\) definida por
    \[
    f(x) = 
    \begin{cases}
        x^2 - 1; & \textrm{se } x \geq 1 \\
        x - 3; & \textrm{se } -1 \leq x < 1 \\
        2x + 1; & \textrm{se } x < -1
    \end{cases}
    \]
    
    \item \textbf{Funções Exponenciais:}
    
    \(f(x) = 2^{x - 1}\)
    
    \item \textbf{Funções Logarítmicas:}
    
    \(f(x) = \log_2 x\)
    
    \item \textbf{Funções Trigonométricas:}
    
    \(f(x) = \cos x\)
    
    \(f(x) = \sin y\)
    
    \(f(x) = \sin \left(x - \frac{\pi}{2}\right)\)
    
    \(f(x) = \sin \left[x - \frac{\pi}{2}\right]\)
    
    \(f(x) = \sin \left\{x - \frac{\pi}{2}\right\}\)
    
\end{enumerate}


\section*{Capítulo 5: Cálculo}

\begin{enumerate}
    \item Limites
    \begin{enumerate}
        \item Calcule os limites abaixo:
        \begin{enumerate}
            \item $\limite_{x \to 1} \frac{x^2 - 1}{x-1}$
        \end{enumerate}
    \end{enumerate}
\end{enumerate}




% Fim do Documento
\end{document}
